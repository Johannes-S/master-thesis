\documentclass{article}
\usepackage[paperheight=400mm,paperwidth=250mm,margin=1cm,heightrounded,footskip=0.5cm]{geometry}

\usepackage{tikz}  % For drawing
\usepackage{hyperref}  % For a clickable table of contents

\usetikzlibrary{calc}  % To compute complex tikz node positions
\usetikzlibrary{graphdrawing}  % To use tikz automatic node placements
\usetikzlibrary{positioning}  % For easier notation of node placements (Used for the layout of the different critcal pair steps)

\usegdlibrary{force}  % To use force based tikz node placement algoritms (spring layout)

% NAMING CONVENTIONS
% graph element: [external node, internal node, selector edge, selector label, hyperedge node, hyperedge tentacle, hyperedge tentacle label]
%
% graph element involvement: [rule1, rule2, both, new]
%                Note: Specifies in which rule application a graph element is involved; only applies to critical pair reports
%
% critical pair step: [joint graph, applied rule 1, applied rule 2, canonical 1, canonical 2]
%                Note: only applies to critical pair reports
% grammar part: [left side, right side]
%                Note: only applies to grammar reports
%
% report element id: [report element 1, report element 2, ...]
%                        Note: A report element can be a critical pair, a grammar rule or a heap configuration report
%
% identifier: node identifier, hyperedge identifier, tentacle identifier, selector identifier
% node identifier: <graph id of the node>
% hyperedge identifier: <graph id of the hyperedge>
% tentacle identifier: <graph id of the hyperedge> <tentacle index>
% selector identifier: <graph id of the source node> <label of selector>
%
% concrete id: [1/rule1/node/1, 1/left side/hyperedge/2, 1/selector edge/1 right, 1/selector label/1 right, 1/hyperedge tentacle/1 2, 1/hyperedge tentacle label/1 2, ...]
%                     Note: Identifies exactly one graph element;
%             Format: "[report element id]/[critical pair step]/[graph element]/[identifier]"   for critical pair report element
%             Format: "[report element id]/[grammar part]/[graph element]/[identifier]"   for grammar report element
%             Format: "[report element id]/[graph element]/[identifier]"   for heap configuration report element
%            Important: [graph element] references nodes just by the word "node" and NOT "external node" or "internal node"

%
% Show "graph id": TODO
% Show "report element id": TODO


% Tikz parameters to investigate
%
% weight (give nodes & edges an "importance" -> meaning depends on drawing algorithm)
% length (set length of edge)
% radius (of circular objects)

% Global options: node distance=1cm

% Section 32.1 Controlling and Configuring Force-Based Algorithms
% iterations=500  (global)
% spring constant=0.01 (per edge or global?)
% electric charge=1  (per node or global -> attraction / repulsion between nodes)
% electric force order=1 (global -> reduce long range forces)
% approximate remote forces=true (global -> dont calculate every interaction)
% coarsen=true  (global -> Joins nodes to a coarse graph initially and then refines) -> Other options to configure this further

% Force based Layout algorithms
% spring layout, spring Hu 2006 layout, spring electrical layout, spring electrical layout', spring electrical Hu 2006 layout, spring electrical Walshaw 2000 layout




% TODO: Define macros

% Global style settings
\tikzset {  % TODO: Add style options that distinguish nodes/edges that are from rule 1 / rule 2 or both
    external node/.style={draw,double,rectangle,rounded corners=5pt},
    internal node/.style={draw,circle,electric charge=1},
    handle node/.style={fill,circle,inner sep=2pt,electric charge=1},
    selector edge/.style={->,thick},
    selector label/.style={midway,sloped,auto=left},
    hyperedge node/.style={draw,rectangle,electric charge=2},
    hyperedge tentacle/.style={black!40},
    hyperedge tentacle label/.style={black,anchor=center,rectangle,rounded corners=4pt,fill=white,inner sep=1pt},
    global heap drawing settings/.style={spring electrical layout,node distance=1.6cm,electric force order=3},
    grammar derivation arrow/.style={},
    node primitive type/.style={fill=gray},
    all reduction tentacle/.style={green},
    none reduction tentacle/.style={red},
    mixed reduction tentacle/.style={blue}
}

\def\AttestorStronglyJoinable{STRONGLYJOINABLE}
\def\AttestorWeaklyJoinable{WEAKLYJOINABLE}

\newcommand{\AttestorJoinabilityResultString}[1]{%
    \ifx#1\AttestorStronglyJoinable
        Strongly Joinable%
    \else
        \ifx#1\AttestorWeaklyJoinable
            Weakly Joinable%
        \else
            Not Joinable%
        \fi
    \fi
}

\newcommand{\AttestorJoinabilityResultSymbol}{%
    \pgfkeys{/attestor/joinability result/.get=\joinabilityResult}%
    \ifx\joinabilityResult\AttestorStronglyJoinable
        $=$%
    \else
        \ifx\joinabilityResult\AttestorWeaklyJoinable
            $\cong$%
        \else
            $\neq$%
        \fi
    \fi
}

\newcommand{\AttestorCriticalPairReport}{
    \pgfkeys{/attestor/joinability result/.get=\joinabilityResult}
    \pgfkeys{/attestor/rule 1/label/.get=\ruleALabel}
    \pgfkeys{/attestor/rule 2/label/.get=\ruleBLabel}

    \edef\criticalPairTitle{Critical Pair (\AttestorJoinabilityResultString{\joinabilityResult})}
    \subsection{\criticalPairTitle}
    
    \begin{tikzpicture}
        \matrix[ampersand replacement=\&, anchor=south west, column sep=1cm, row sep=1mm] (matrix) {
            \node[anchor=north] (appliedR1) {\tikz{\AttestorDrawHeapConfiguration{/attestor/applied rule 1}}}; \&
            \node[anchor=north] (appliedR2) {\tikz{\AttestorDrawHeapConfiguration{/attestor/applied rule 2}}}; \\
            \node[anchor=center,rotate=90,inner sep=2pt] (arrow1) {\LARGE $\Rightarrow$}; \&
            \node[anchor=center,rotate=270,inner sep=2pt] (arrow2) {\LARGE $\Leftarrow$}; \\
            \node[anchor=center] (canonical1) {\tikz{\AttestorDrawHeapConfiguration{/attestor/canonical 1}}}; \&
            \node[anchor=center] (canonical2) {\tikz{\AttestorDrawHeapConfiguration{/attestor/canonical 2}}}; \\
        };
        \node[anchor=south] (jointGraph) at ($ (appliedR1.north)!.5!(appliedR2.north) + (0,1.5cm)$) {\tikz{\AttestorDrawHeapConfiguration{/attestor/joint graph}}};
        \path (appliedR1.north) -- node[sloped] (arrow3) {\LARGE $\Rightarrow$} (jointGraph.south);
        \path (appliedR2.north) -- node[sloped] (arrow4) {\LARGE $\Leftarrow$} (jointGraph.south);
        \path (canonical1) -- node[sloped, allow upside down] {\LARGE \AttestorJoinabilityResultSymbol} (canonical2);
        \node[inner sep=0pt,anchor=north east] at (arrow1.north east)  {\large $*$};
        \node[inner sep=0pt,anchor=north west] at (arrow2.north west)  {\large $*$};
        \node[inner sep=1pt,anchor=south east] at (arrow3.north)  {\ruleALabel};
        \node[inner sep=1pt,anchor=south west] at (arrow4.north)  {\ruleALabel};
        
    \end{tikzpicture}

    % TODO Table with debug information
    %\begin{tabular}{ r | l l | l l }
    %    Node & Type R1 & Type R2 & Red. tent. R1 & Red. tent. R2 \\
    %    \hline
    %    1 & NULL  & NULL  & N/A  & False \\
    %    2 & type1 & type1 & True & False \\
    %    3 & NULL  & type1 & N/A  & False \\
    %    4 & type2 & NULL  & N/A  & True  \\
    %    5 & NULL  & NULL  & N/A  & False \\
    %\end{tabular}
}

\newcommand{\AttestorGrammarReport}{
    \def\true{true}
    \pgfkeys{/attestor/is original rule/.get=\isOriginalRule}
    \pgfkeys{/attestor/original rule idx/.get=\originalRuleIdx}
    \ifx\isOriginalRule\true
        \edef\grammarTitle{Grammar Rule \originalRuleIdx}
        \subsection{\grammarTitle}
    \else
        \pgfkeys{/attestor/collapsed rule idx/.get=\collapsedRuleIdx}
        \edef\grammarTitle{Rule \originalRuleIdx.\collapsedRuleIdx}
        \subsubsection{\grammarTitle}
    \fi

    \begin{tikzpicture}
        \node (lhs) {\tikz{\AttestorDrawHeapConfiguration{/attestor/left hand side}}};
        \node[right=0cm of lhs] (arrow) {\LARGE $\Rightarrow$};
        \node[right=0cm of arrow] (rhs) {\tikz{\AttestorDrawHeapConfiguration{/attestor/right hand side}}};
    \end{tikzpicture}

}

\newcommand{\AttestorHeapConfigurationReport}{
    \section{Heap Configuration}

    \tikz{\AttestorDrawHeapConfiguration{/attestor}}
}

% #1: Path to current heap  #2: current node id
\newcommand{\AttestorDrawNode}[2] {
    \def\true{true}
    \def\currentNodePath{#1/nodes/#2}
    % Check if the node type is primitive
    \pgfkeys{\currentNodePath/is primitive/.get=\isPrimitive}
    \tikzset{temp style/.code={
        \ifx\isPrimitive\true
            \tikzset{node primitive type}
        \fi
    }}
    \def\pathLHS{/attestor/left hand side}
    \def\pathRHS{/attestor/right hand side}
    \def\currentHeapPath{#1}
    \def\pathJointGraph{/attestor/joint graph}
    \pgfkeys{\currentNodePath/is external/.get=\isExternal}
    \ifx\isExternal\true
        % Is external node
        \ifx\currentHeapPath\pathLHS
            % Is node in LHS of grammar rule
            \node[handle node] (node#2) {};
        \else
            \pgfkeys{\currentNodePath/external indices/.get=\externalIdices}
            \ifx\currentHeapPath\pathRHS
                % Is node in RHS of grammar rule
                \node [external node,temp style] (node#2) {$(\externalIdices)$};
            \else
                % Is normal external node
                %\edef\nodeStyles{[external node] (node#2) {$#2(\externalIdices)$};}
                \node [external node,temp style] (node#2) {$#2(\externalIdices)$};
            \fi
        \fi
    \else
        \ifx\currentHeapPath\pathJointGraph
            % Is node in joint graph of a critical pair
            %\def\nodeStyles{[internal node] (node#2) {#2};}
            \node [internal node,temp style] (node#2) {#2};
            %\node[external node] (node#2) {$#2^{(?)}_{(?)}$}; % TODO
        \else
            % normal internal node
            %\edef\nodeStyles{[internal node]}
            \node [internal node,temp style] (node#2) {#2};
        \fi

    \fi
    % TODO: Select correct case and get the values of \id, \externalHC1, \externalHC2, \external and apply correct styles
    % Joint Heap Configuration, external in both rules
    %\node[external node] (node\id) {${\id}^{(\externalHC1)}_{(\externalHC2)}$};
    % Joint Heap Configuration, external in rule 1
    %\node[external node] (node\id) {${\id}^{(\externalHC1)}$};
    % Joint Heap Configuration, external in rule 2
    %\node[external node] (node\id) {${\id}_{(\externalHC2)}$};
}

% #1 is the pgfkeys prefix. The macro must be called from inside a tikzpicture environment
\newcommand{\AttestorDrawHeapConfiguration}[1] {
    \edef\all{all}
    \edef\none{none}
    \def\true{true}
    \pgfkeys{#1/nodes/.get=\nodelist}
    \pgfkeys{#1/nonterminals/.get=\nonterminalList}
    % 1. Rendering pass: Only draw nodes correctly for the node placement algorithm (otherwise there is an issue with the edges)
    \begin{scope}[global heap drawing settings]
        % 1. draw nodes
        \foreach \n in \nodelist {
            \AttestorDrawNode{#1}{\n}
        }

        % 2. draw nonterminals
        \foreach \nonterminal in \nonterminalList {
            \pgfkeys{#1/nonterminals/\nonterminal/label/.get=\nonterminalLabel}
            \node [hyperedge node] (nonterminal\nonterminal) {$^{\nonterminal}$\nonterminalLabel};
            \pgfkeys{#1/nonterminals/\nonterminal/tentacle targets/.get=\targets}
            \foreach \target in \targets {
                \pgfkeys{#1/nonterminals/\nonterminal/tentacles/\target/indices/.get=\labels}
                \pgfkeys{#1/nonterminals/\nonterminal/tentacles/\target/reduction tentacle/.get=\reductionTentacle}
                \edef\tentacleStyle{%
                    \ifx\reductionTentacle\all
                        all reduction tentacle%
                    \else
                        \ifx\reductionTentacle\none
                            none reduction tentacle%
                        \else
                            mixed reduction tentacle%
                        \fi
                    \fi
                }
                \edef\DrawLine{(nonterminal\nonterminal) edge [hyperedge tentacle,\tentacleStyle] node [hyperedge tentacle label] {\labels} (node\target);}
                \draw \DrawLine
            }
        }

        % 3. draw selector edges
        \foreach \source in \nodelist {
            \pgfkeys{#1/nodes/\source/selector targets/.get=\targets}
            \foreach \target in \targets {
                \pgfkeys{#1/nodes/\source/selectors/\target/labels/.get=\labels}
                \ifx\source\target
                    \edef\DrawLine{(node\source) edge [selector edge,loop above] node [selector label] {\labels} (node\target);}
                \else
                    \pgfkeys{#1/nodes/\source/selectors/\target/has reverse/.get=\hasReverse}
                    \ifx\hasReverse\true
                        \edef\DrawLine{(node\source) edge [selector edge, bend left] node [selector label] {\labels} (node\target);}
                    \else
                        \edef\DrawLine{(node\source) edge [selector edge] node [selector label] {\labels} (node\target);}
                    \fi
                \fi
                \draw \DrawLine
            }
        }
    \end{scope}

}
