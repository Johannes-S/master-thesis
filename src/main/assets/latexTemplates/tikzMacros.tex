\documentclass{article}
\usepackage{tikz}
\usetikzlibrary{graphdrawing}
\usegdlibrary{force}

% NAMING CONVENTIONS
% graph element: [external node, internal node, selector edge, selector label, hyperedge node, hyperedge tentacle, hyperedge tentacle label]
%
% graph element involvement: [rule1, rule2, both, new]
%                Note: Specifies in which rule application a graph element is involved; only applies to critical pair reports
%
% critical pair step: [joint graph, applied rule 1, applied rule 2, canonical 1, canonical 2]
%                Note: only applies to critical pair reports
% grammar part: [left side, right side]
%                Note: only applies to grammar reports
%
% report element id: [report element 1, report element 2, ...]
%                        Note: A report element can be a critical pair, a grammar rule or a heap configuration report
%
% identifier: node identifier, hyperedge identifier, tentacle identifier, selector identifier
% node identifier: <graph id of the node>
% hyperedge identifier: <graph id of the hyperedge>
% tentacle identifier: <graph id of the hyperedge> <tentacle index>
% selector identifier: <graph id of the source node> <label of selector>
%
% concrete id: [1/rule1/node/1, 1/left side/hyperedge/2, 1/selector edge/1 right, 1/selector label/1 right, 1/hyperedge tentacle/1 2, 1/hyperedge tentacle label/1 2, ...]
%                     Note: Identifies exactly one graph element;
%             Format: "[report element id]/[critical pair step]/[graph element]/[identifier]"   for critical pair report element
%             Format: "[report element id]/[grammar part]/[graph element]/[identifier]"   for grammar report element
%             Format: "[report element id]/[graph element]/[identifier]"   for heap configuration report element
%            Important: [graph element] references nodes just by the word "node" and NOT "external node" or "internal node"

%
% Show "graph id": TODO
% Show "report element id": TODO


% TODO: Define macros


% TODO: Don't include the document part, because this will be added by the java code
\begin{document}

\section{Critical Pair}  % TODO: Enumerate the critical pairs

Strongly joinable  % TODO: Replace with joinability result
\label{sec:criticalpair:1}

% Global style settings
\tikzset {  % TODO: Add style options that distinguish nodes/edges that are from rule 1 / rule 2 or both
    external node/.style={draw,circle,double},
    internal node/.style={draw,circle},
    selector edge/.style={red,->,thick},
    selector label/.style={auto},
    hyperedge node/.style={draw,green,rectangle},
    hyperedge tentacle/.style={blue,thick},
    hyperedge tentacle label/.style={auto}
}

% Graph drawing
\begin{tikzpicture}[spring layout]
    \begin{scope}[]
        % Nodes
        \node [external node] (node0) {0};
        \node [internal node] (node1) {1};
        \node [internal node] (node2) {2};
        \node [internal node] (node3) {3};
        % Hyperedges
        \node [hyperedge node] (hyperedgeRefBT1) {RefBT};
        \node [hyperedge node] (hyperedgeRefBT2) {RefBT};
        % Selector edges
        \draw (node1) edge [selector edge] node [selector label] {left} (node2);
        \draw (node1) edge [selector edge] node [selector label] {right} (node3);
        % Tentacle-Edges
        \draw (hyperedgeRefBT1) edge [hyperedge tentacle] node [hyperedge tentacle label] {0} (node2);
        \draw (hyperedgeRefBT1) edge [hyperedge tentacle] node [hyperedge tentacle label] {1} (node0);
        \draw (hyperedgeRefBT2) edge [hyperedge tentacle] node [hyperedge tentacle label] {0} (node3);
        \draw (hyperedgeRefBT2) edge [hyperedge tentacle] node [hyperedge tentacle label] {1} (node0);
    \end{scope}
    % TODO: Add the other abstracted graphs
\end{tikzpicture}

% Table with debug information
\begin{tabular}{ r | l l | l l }
    Node & Type R1 & Type R2 & Red. tent. R1 & Red. tent. R2 \\
    \hline
    1 & NULL  & NULL  & N/A  & False \\
    2 & type1 & type1 & True & False \\
    3 & NULL  & type1 & N/A  & False \\
    4 & type2 & NULL  & N/A  & True  \\
    5 & NULL  & NULL  & N/A  & False \\
\end{tabular}

    


\end{document}